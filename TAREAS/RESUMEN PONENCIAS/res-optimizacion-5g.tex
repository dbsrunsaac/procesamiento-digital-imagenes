\documentclass[12pt,a4paper]{article}

% Paquetes básicos
\usepackage[spanish]{babel}
\usepackage[utf8]{inputenc}
\usepackage[T1]{fontenc}
\usepackage{geometry}
\usepackage{graphicx}
\usepackage{hyperref}
\usepackage{enumitem}

% Márgenes
\geometry{margin=2.5cm}

% Datos del resumen
\title{Resumen: \textbf{Optimización de recursos en redes 5G para la prestación de servicios eMMB y URLLC}}

\author{Davis Bremdow Salazar Roa}
\date{29 de agosto de 2025}

\begin{document}
	
	\maketitle
	
	\section{Introducción}
	
	En un entorno competitivo la ponencia destaco la importancia de los diferentes modelos de negocio (diferenciado y sin diferenciar) analizando para ello la calidad percibida por cada modelo y la existencia de diferentes entidades que compiten entre si para obtener el mayor beneficio según se define en la teoría de juegos.
	
	\section{Desarrollo}
	
	En la creciente demanda de conectividad y calidad de servicio que se ve afectada por la cantidad de usuarios en aumento y tecnologías de reciente auge como la conducción automática y operaciones medicas a larga distancia requieren de ciertos niveles de calidad en la conectividad que requieren de una respuesta rápida o de baja latencia (URLLC), mientras que en el otro lado de la moneda se contemplan servicios que no requieren de forma crítica una baja latencia si no que estos mantengan una estabilidad sin importar la velocidad de respuesta (eMMB).
	
	Es así que bajo tal propuesta del mercado actual en las comunicaciones 5G se proponen diferentes modelos de negocio que permitan evaluar el desempeño de cada uno utilizando como variable para tal propósito la calidad en un servicio y la probabilidad de que uno sea escogido en función a sus atributos, beneficios fundamentando los esquemas matemáticos mediante la teoría de juegos.
	
	\section{Conclusiones}
	De los modelos propuestos se deriva los siguiente:
	\begin{itemize}
		\item El mejor mercado una gran satisfacción en los clientes son los monopolios
		\item Es mejor y más provechoso para el público una organización con oferta de varios servicios elevando la calidad en su servicio y satisfacción general
	\end{itemize}
	
	\section{Observaciones}
	A pesar del gran beneficio que ofrecen los monopolios en el modelo planteado este en la realidad no es aplicable o es ideal debido a que la competencia surge como un ente natural y el cual entra en conflicto con algún modelo establecido y el cual a su vez trata de brindar un mejor servicio tratando de obtener beneficios en función a ello.
\end{document}


































