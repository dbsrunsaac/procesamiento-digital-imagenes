\documentclass[12pt,a4paper]{article}

% Paquetes básicos
\usepackage[spanish]{babel}
\usepackage[utf8]{inputenc}
\usepackage[T1]{fontenc}
\usepackage{geometry}
\usepackage{graphicx}
\usepackage{hyperref}
\usepackage{enumitem}

% Márgenes
\geometry{margin=2.5cm}

% Datos del resumen
\title{Resumen: \textbf{Medición del área del micelio cultivado en placas Petri mediante un sistema de procesamiento de imágenes}}

\author{Davis Bremdow Salazar Roa}
\date{29 de agosto de 2025}

\begin{document}
	
	\maketitle
	
	\section{Introducción}
	El micelio es un hongo utilizado para el cultivo de bacterias y cuyo crecimiento cuya dependencia se encuentra en función al área que ocupa, el cuidado y desarrollo de bacterias es de vital importancia en áreas como la biología en la cual se mantiene una población de estas para su estudio, es así que para optimizar este proceso en el cálculo del área se propone un modelo para el procesamiento de imágenes que permita obtener de forma precisa tal valor.
	
	\section{Desarrollo}
	La problemática principal surge de las áreas irregulares y la importancia en la evolución y extensión de las infecciones fungicas debido al crecimiento de los hongos, es así que para validar y estudiar posibles causas comprender su crecimiento es de vital importancia, el método inicial planteado en los laboratorios a nivel nacional consta de un proceso manual que aunque efectivo esta sujeto a errores y con lleva un tiempo considerable en la medición y calibración de herramientas, por lo tanto se plantea el uso de herramientas digitales para la medición del área de micelio mediante procesamiento de imágenes superando varias dificultades como la no homogeneidad y de perspectiva aplicando para su solución diferentes filtros y métodos para la transformación de la imagen. 
	
	\section{Conclusiones}
	Del método propuesto se infiere:
	\begin{itemize}
		\item Calculo del área del micelio con un certeza del 97\% y el cual se corresponde con su validación manual
		\item Se aplicaron algoritmos de filtrado y segmentación para la detección del micelio y se desarrollo una interfaz gráfica que facilita este proceso.
	\end{itemize}
	
	\section{Observaciones}
	Se encontraron limitantes para imágenes con bajo contraste (bajo contenido micelial) así como limitantes en la ejecución del algoritmo sujeta al hardware siendo recomendable su aplicación en un Rasperry Pi para acelerar tal proceso.
\end{document}


































