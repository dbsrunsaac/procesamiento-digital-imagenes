\usepackage[spanish, es-ucroman]{babel}
\usepackage[utf8]{inputenc}
\usepackage{fancyhdr}
\usepackage{graphicx}
\usepackage[inner=3.0cm,outer=2.5cm,top=3cm,bottom=3cm]{geometry}%margen 2.54
\usepackage{hlundef}
\usepackage{portada}
\usepackage[printonlyused]{acronym}
\usepackage{breakcites}
\usepackage{color,soul}
\usepackage{ae}
\usepackage[usenames,dvipsnames,table,xcdraw]{xcolor}
\usepackage{amsmath}
\usepackage{multirow}
\usepackage[inline]{enumitem}
\usepackage{tikz,pgfplots,pgfplotstable}
\usepackage{capt-of}
\usepackage{csvsimple}
\usepackage[breaklinks]{hyperref}
\setlength{\parindent}{1.27cm}
\usepackage[acronym]{glossaries} % paquete de acrónimos
\usepackage{caption}
%\usepackage{subcaption}
\usepackage{booktabs}
\usepackage{enumitem}
\usepackage[colorinlistoftodos]{todonotes}
\usepackage{mathrsfs}
\newtheorem{proposicion}{Proposición}[section]
\usepackage{amsfonts}
\usepackage{etoolbox}
\usepackage[group-separator={,}]{siunitx}
\usepackage{graphicx} 
\usepackage{adjustbox} % también para rotar
\usepackage{array} % Para definir ancho tabla
\usepackage{booktabs}
\newcommand{\ra}[1]{\renewcommand{\arraystretch}{#1}}
\setcounter{secnumdepth}{3}
\pgfplotsset{compat=1.13}
\usepgfplotslibrary{colormaps}
\usepackage{sidecap}
\sidecaptionvpos{figure}{b}
\sidecaptionvpos{table}{b}
\usepackage{natbib}
\setcitestyle{round}
\usepackage{tikz-qtree}
\usepackage{arydshln}
\usetikzlibrary{shapes.geometric, arrows, decorations.pathreplacing, shadows,trees, positioning, calc, tikzmark, spy, mindmap, matrix}
\tikzstyle{roundedFrame} = [rectangle, text centered, rounded corners, draw=gray]
\tikzstyle{input} = [rectangle,text centered, draw=none]
\tikzstyle{emptyFrame} = [rectangle,text centered, draw=white]
\tikzstyle{frame} = [rectangle, text centered, draw=gray]
\tikzstyle{arrow} = [thick,->,>=stealth]
\usepackage{pifont} % DING
% código para sacar el encabezado de chapter
\usepackage[T1]{fontenc}
\usepackage{titlesec, blindtext, color}
\definecolor{gray75}{gray}{0.75}
\newcommand{\hsp}{\hspace{20pt}}
\titleformat{\chapter}[hang]{\Huge\bfseries}{\thechapter\hsp\textcolor{gray75}{|}\hsp}{0pt}{\Huge\bfseries}
% Nuevos comandos para diagrama de gantt
% A package which allows simple repetition counts and some useful commands
\usepackage{forloop}
\newcounter{loopcntr}
\newcommand{\rpt}[2][1]{%
  \forloop{loopcntr}{0}{\value{loopcntr}<#1}{#2}%
}
\newcommand{\on}[1][1]{
  \forloop{loopcntr}{0}{\value{loopcntr}<#1}{&\cellcolor{gray75}}
}
\newcommand{\off}[1][1]{
  \forloop{loopcntr}{0}{\value{loopcntr}<#1}{&}
}
\usepackage[nottoc]{tocbibind}
