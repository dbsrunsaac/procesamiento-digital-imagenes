\chapter{Marco Teórico}
\section{Líquenes}
Los líquenes son organismos compuestos que resultan de la asociación simbiótica entre un hongo y un alga o cianobacteria. Esta relación permite que los líquenes se adapten a ambientes extremos y colonizen superficies expuestas, como los muros y monumentos históricos. Su importancia radica en su capacidad para sobrevivir en condiciones adversas, lo que los convierte en uno de los principales agentes biológicos en el biodeterioro de los materiales pétreos.

En el contexto del estudio de daños en patrimonio histórico, los líquenes pueden provocar tanto efectos químicos como físicos sobre la piedra. La secreción de sustancias ácidas o enzimas, junto con la penetración de sus estructuras en la roca, ocasiona la degradación y alteración de la superficie, afectando su resistencia y estética. Por esto, su análisis es fundamental para la conservación y protección del patrimonio.

\section{Cobertura Liquenica}
La cobertura líquenica se refiere al porcentaje o área de la superficie del muro o monumento que está colonizada por líquenes. Medir esta cobertura permite cuantificar la presencia de estos organismos y evaluar la magnitud del riesgo que representan para la integridad estructural y estética del patrimonio.

Este concepto es crucial para investigaciones que buscan identificar zonas más vulnerables al biodeterioro, facilitando así el diseño de intervenciones o tratamientos específicos. La cobertura líquenica también ayuda a monitorear el avance o retroceso de la colonización en distintos períodos temporales, contribuyendo a la evaluación de la efectividad de las estrategias de conservación.

\section{Simbiosis liquénica}
La simbiosis liquénica es la relación mutualista entre un hongo y un alga o cianobacteria que conforman el liquen. En esta relación, el hongo proporciona estructura y protección, mientras que el alga o cianobacteria realiza la fotosíntesis para suministrar nutrientes. Esta asociación es clave para la supervivencia en hábitats donde otros organismos vegetales no podrían crecer.

Esta simbiosis permite que los líquenes se establezcan en superficies inertes como la piedra, donde desarrollan un impacto significativo en el deterioro del patrimonio histórico. Entender esta interacción es relevante para comprender cómo actúan los líquenes y qué mecanismos biológicos permiten su persistencia y daño sobre las superficies monumentales.

\section{Biodeterioro}
El biodeterioro describe el proceso de degradación o daño a materiales causado por organismos vivos, incluyendo líquenes, bacterias, hongos y plantas. Este fenómeno es un factor clave en el deterioro de muros y monumentos históricos, ya que contribuye a la pérdida de propiedades mecánicas, químicas y visuales del material pétreo.

Estudiar el biodeterioro biomédico permite identificar los mecanismos de daño específicos y las especies responsables, además de evaluar las condiciones que favorecen su desarrollo. Esto es fundamental para implementar medidas preventivas y correctivas en la conservación del patrimonio, minimizando el impacto biológico sobre estructuras valiosas.
\section{Patrimonio arqueológico}
El patrimonio arqueológico comprende todos los bienes culturales y monumentos que poseen valor histórico, artístico o científico, resultantes de culturas pasadas. La conservación de este patrimonio es vital para preservar la memoria y la identidad cultural de la humanidad, así como para su estudio y conocimiento.

La investigación sobre el daño que los líquenes y otros agentes biológicos causan a este patrimonio es esencial para desarrollar estrategias de protección adecuadas. La presencia de líquenes puede acelerar el deterioro de las estructuras arqueológicas, comprometiendo su estabilidad física y su valor histórico, lo que hace indispensable su monitoreo y estudio.
\section{Deterioro pétreo}
El deterioro pétreo es el proceso mediante el cual las piedras que forman muros o monumentos sufren alteraciones físicas, químicas o biológicas que afectan su estructura y apariencia. Este deterioro puede reducir la vida útil del monumento y deteriorar su valor cultural y estético.

En relación con los líquenes, el deterioro pétreo se manifiesta por la acción combinada de la penetración de sus estructuras, la secreción de ácidos y compuestos orgánicos, y la retención de humedad. Estas acciones favorecen la fragmentación, pérdida de cohesión y alteración química de la piedra, lo que hace prioritario su estudio para salvaguardar los monumentos históricos.
\section{Segmentación de imágenes}
La segmentación de imágenes es una técnica de procesamiento digital que consiste en dividir una imagen en regiones o segmentos que representan objetos específicos, como las áreas cubiertas por líquenes sobre un muro. Esta técnica facilita el análisis cuantitativo y visual del daño biológico en superficies patrimoniales.

En investigaciones sobre el efecto de los líquenes, la segmentación permite identificar con precisión la extensión y distribución de la cobertura líquenica en fotografías o imágenes digitales. Esto es importante para evaluar el avance del biodeterioro y planificar intervenciones de conservación basadas en datos objetivos y reproducibles.
\section{Reflectancia espectral}
La reflectancia espectral mide la cantidad de luz que una superficie refleja en función de la longitud de onda. Esta propiedad es utilizada para caracterizar los materiales y organismos presentes en el estudio, diferenciando entre tipos de líquenes y estados de conservación sin contacto físico directo.

En investigaciones de deterioro de monumentos, la reflectancia espectral se emplea para realizar análisis no invasivos que permiten detectar y monitorear la presencia y salud de los líquenes. Esta tecnología contribuye a obtener datos precisos para el diagnóstico y la toma de decisiones en conservación.
\section{Índices espectrales}
Los índices espectrales son fórmulas que combinan valores de reflectancia a diferentes longitudes de onda para resaltar características específicas de las superficies, como la presencia de materia orgánica o humedad. Son herramientas potentes para distinguir líquenes y evaluar su impacto en materiales pétreos.

El uso de índices espectrales en la investigación facilita la clasificación y cuantificación de la cobertura líquenica, mejorando la detección temprana del biodeterioro. Estos índices permiten la integración de datos remotos con observaciones de campo para un análisis más completo y eficiente.

\section{Validación de campo}
La validación de campo consiste en confrontar y corroborar, directamente en el sitio, los datos obtenidos por métodos remotos o analíticos con observaciones y mediciones reales. Este paso es imprescindible para asegurar la precisión y aplicabilidad de los resultados en estudios de deterioro y cobertura líquenica.

La validación permite ajustar y mejorar las técnicas de análisis, como la segmentación de imágenes o el uso de índices espectrales, asegurando que reflejen correctamente la realidad del estado de los monumentos. Además, garantiza que las recomendaciones de conservación estén basadas en información verificada y confiable.


