\chapter{Antecedentes}

\section{Trabajos Relacionados}
\subsection{“Biota liquénica en el monumento arqueológico de Pawkarkancha Santuario Histórico de Machupicchu”}\
Este estudio realizado por investigadores de la Universidad Nacional de San Antonio Abad del Cusco publicado en el 2022, analizó la colonización de hongos liquenizados en muros arqueológicos del sitio de Pawkarkancha, ubicado en los valles Q’esqa y Pampacahuana. Empleó el método modificado de Mostacedo \& Fredericksen (2000) para cuantificar la biota. Este trabajo constituye un referente metodológico relevante para investigaciones sobre biodiversidad liquénica en estructuras patrimoniales de piedra.\cite{molina2022biota}.
\subsection{“Inhibición del crecimiento de líquenes y musgos en elementos líticos usados en la construcción de fachadas de iglesias patrimoniales cusqueñas aplicando nanoburbujas de aire”}\
Este trabajo en conjunto realizado por investigadores de la Universidad Agraria de la Molina y la Universidad Andina del cusco, publicado en el 2021, desarrollaron un diseño metodológico aplicado, descriptivo y explicativo en dos etapas, con el fin de evaluar la eficacia de las nanoburbujas de aire en la eliminación de patologías bióticas. En la primera etapa se caracterizaron las patologías mediante un muestreo no probabilístico de los elementos arquitectónicos de las fachadas de la Basílica Catedral y la Iglesia de la Compañía de Jesús, utilizando fichas de observación y mapeo digital. Se determinó que el 75\% de las rocas presentaban patologías, predominando las biológicas.

En la segunda etapa experimental se aplicaron nanoburbujas de aire a 20 especímenes líticos provenientes de las canteras de Rumicolca, Huacoto y Sacsayhuamán, con diferentes tiempos de exposición (10, 20 y 30 minutos). Los resultados demostraron que el tratamiento de 30 minutos logró eliminar completamente líquenes y musgos sin dañar el sustrato pétreo, consolidándose como una técnica reproducible, ecológica y eficaz para la conservación del patrimonio lítico.\cite[]{valverde2021inhibicion}
\subsection{Los líquenes y la degradación del patrimonio arquitectónico}\
En este estudio se aborda la influencia de los líquenes en los procesos de degradación y conservación del patrimonio arquitectónico. Estos organismos, resultado de la simbiosis entre hongos y algas, poseen una alta capacidad de colonización sobre superficies pétreas, incluidas aquellas de valor histórico. Los líquenes saxícolas, tanto epilíticos como endolíticos, generan biodeterioro a través de mecanismos mecánicos como la penetración y disgregación del sustrato, los me químicos, mediante la liberación de ácidos orgánicos, entre ellos el ácido oxálico, que modifica la composición mineral de la roca.

La eliminación de líquenes requiere una evaluación técnica cuidadosa, ya que su remoción puede dejar la piedra expuesta y favorecer una recolonización más rápida. Entre los métodos de limpieza empleados se incluyen la remoción mecánica manual, el hidrolavado y la aplicación de biocidas (como el hipoclorito de litio), aunque estos últimos pueden afectar la integridad del material. Por ello, las estrategias de conservación se complementan con tratamientos de consolidación e hidrofugación, destinados a reforzar la cohesión interna de la piedra y disminuir su porosidad.\cite[Gamboa Osorio et al., 2017]{Gamboa2017}.
\subsection{“Determinación de la diversidad de líquenes saxícolas de tres sitios arqueológicos de Cajamarca”}
El estudio realizado en el año 2016 tuvo como objetivo evaluar la diversidad y abundancia de líquenes saxícolas en tres sitios arqueológicos del departamento de Cajamarca: Cumbemayo, Santa Apolonia y Ventanillas de Otuzco. La recolección de muestras se efectuó mediante espátulas y cinceles metálicos, asegurando la obtención de ejemplares en buen estado, los cuales fueron almacenados en sobres de papel secante, registrando previamente sus características morfológicas y del hábitat. La identificación taxonómica se realizó a través de claves especializadas y observación microscópica de estructuras reproductivas.

Para el análisis poblacional, se aplicaro parámetros de densidad poblacional (organismos/m²), densidad relativa y dominancia, empleando el método del cuadrado para la estimación cuantitativa de la población liquénica.Esto permitio identificar 19 especies de líquenes saxícolas, destacando Cumbemayo como el sitio con mayor diversidad (13 especies), seguido de Ventanillas de Otuzco (5 especies) y Santa Apolonia (3 especies).\cite{MarinoValle2016}.
\subsection{“Macro-líquenes del Santuario Histórico de Machu Picchu”}\
En 2015 realizaron el estudio con el objetivo de documentar y actualizar la liquenobiota de esta zona patrimonial del Cusco, aplicando una metodología tradicional de recolección e identificación.

El muestreo se efectuó en cuatro estaciones georreferenciadas —sitio ceremonial, subida a la montaña Machu Picchu, camino al puente Inca y alrededores del Inka Terra Hotel. Se recolectaron muestras en diferentes sustratos (rocas, troncos y cortezas), bajo autorización de la Resolución Directoral N.º 009-2012-AG-DGFFS-DGEFFS.Los métodos aplicados, lograron nuevos registros para el Perú y aportando una base científica para futuros estudios ecológicos y de conservación en los Andes peruanos.\cite{nunez2015nuevos}

