\chapter{Hipótesis}

\section{Hipótesis general}

El análisis digital de imágenes de líquenes sobre rocas incaicas permite identificar patrones cuantificables de deterioro superficial, posibilitando la detección temprana de daños y la implementación de medidas de alerta para realizar acciones preventivas efectivas por parte del personal encargado en los centros arqueológicos.

\section{Hipótesis específicas}
\begin{enumerate}
    \item Las técnicas de segmentación y clasificación de imágenes permiten distinguir de manera precisa las áreas cubiertas por líquenes en superficies pétreas incaicas.
    \item Existe una correlación significativa entre la densidad de líquenes y el grado de alteración superficial de las rocas.
    \item El uso de modelos digitales de análisis contribuye a la toma de decisiones en la conservación preventiva del patrimonio arqueológico.
\end{enumerate}
