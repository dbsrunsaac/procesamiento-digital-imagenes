\chapter{Justificación}\
Los muros incaicos representan una herencia cultural  invaluable, cuya preservación ha sido un desafío constante. Durante décadas  la detección y limpieza de líquenes en estas estructuras se realiza de manera tradicional, dependiendo del criterio y la experiencia de los encargados de mantenimiento. Sin embargo, la  mala extracción  de los líquenes puede generar consecuencias perjudiciales, como el crecimiento de otros microorganismos, debido a que las superficies expuestas y las alteraciones en la composición del geomaterial facilitan su crecimiento.

Ante este problema, el presente proyecto propone una herramienta tecnológica no invasiva basada en el procesamiento digital de imágenes, que permita identificar, cuantificar y monitorear la presencia de líquenes en superficies pétreas. Esta técnica contribuye a la conservación preventiva del patrimonio cultural, al ofrecer una evaluación precisa, continua.

La propuesta busca superar las limitaciones de los métodos visuales tradicionales, aportando mayor precisión, escalabilidad y capacidad de análisis, con el fin de establecer estrategias más efectivas de mantenimiento y control del crecimiento de líquenes en parques arqueológicos.

