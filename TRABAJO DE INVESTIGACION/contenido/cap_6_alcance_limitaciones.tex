\chapter{Alcances y Limitaciones}

\section{Alcance}
El presente estudio se centra en el análisis de imágenes digitales de líquenes que crecen sobre estructuras líticas incaicas, particularmente en zonas arqueológicas seleccionadas del Cusco.  
Se abordará el uso de técnicas de procesamiento y análisis de imágenes para identificar la extensión y tipo de colonización biológica, con el propósito de aportar información útil para la conservación del patrimonio.  
Los resultados permitirán establecer indicadores visuales y cuantitativos de deterioro, aplicables en programas de monitoreo y gestión patrimonial.

\section{Limitaciones}

\begin{itemize}
    \item El estudio se limita a la identificación visual de líquenes mediante imágenes, sin incluir análisis microbiológicos o químicos de laboratorio.
    \item Las condiciones de iluminación, clima y resolución de las imágenes pueden afectar la precisión del reconocimiento de líquenes.
    \item La investigación se circunscribe a un número limitado de sitios y muestras, por lo que los resultados no pueden generalizarse a todo el patrimonio incaico.
    \item El modelo propuesto se centra en la detección temprana de deterioro, pero no aborda directamente procesos de restauración física.
    \item El reconocimientos de los líquenes se limita a estructuras pétreas uniformes sin deformaciones pronunciadas como las que se esperarian en los monumentos. 
\end{itemize}