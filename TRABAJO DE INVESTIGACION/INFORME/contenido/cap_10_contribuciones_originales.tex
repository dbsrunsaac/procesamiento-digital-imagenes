\chapter{Contribuciones Originales Esperadas}

En el aspecto general de la investigación planteada el estado del arte relacionado destacada métodos de extracción, toma de muestras físicas para la identificación de los diferentes organismos en los líquenes para comprender su interacción y posibles efectos sobre el patrimonio pétreo o incluso en investigaciones pioneras se busca la implementación y uso de los rayos x para realizar investigaciones menos invasivas para el estudio en el campo de la petrología, sin embargo a priori un análisis desde el punto de vista digital mediante el procesamiento de imágenes no se contempla como punto de partida para una solución es así que un procedimiento en función al análisis espectral sería un aporte no aplicado en la actual problemática y que puede ser punto de partida para futuras investigaciones en el campo creciente del procesamiento digital que cada vez cuenta con más y mejores equipos para analizar con mejor detalle incluso longitudes de onda no perceptibles por el ojo humano en el rango del ultravioleta e infrarrojo como se viene aplicando con las cámaras hiperespectrales y multiespectrales.


Las principales contribuciones de este trabajo pueden resumirse en:

 \begin{itemize}
 	
	\item Análisis espectral de las especies simbiontes mediante el procesamiento de imágenes
	
	\item Propone el uso de un espacio limitado al RGB para el procedimiento limitando costos de implementación.
	
	\item Algoritmo de procesamiento basado en los diferentes modos de color y el reconocimiento del área afectada de forma inteligente.
	
 \end{itemize}
